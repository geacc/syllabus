%%%%%%%%%%%%%%%%%%%%%%%%%%%%%%%%%%%%%%%%%
% Inzane Syllabus Template
% LaTeX Template
% Version 1.2 (8.22.2019)
%
% This template has been downloaded from:
% http://www.LaTeXTemplates.com
%
% Original author:
% Carmine Spagnuolo (cspagnuolo@unisa.it) with major modifications by 
% Zane Wolf (zwolf.mlxvi@gmail.com)
%
% I (Zane) have left a lot of instructions both in the .tex file and
% the .cls file that can guide you to customize this document to suite
% your tastes and requirements. Here is a brief guide: 
%  - Changing the Main Color: .cls line 39
%  - Adding more FAQs: .cls line 126 and .tex line 99
%  - Adding TA emails: uncomment .cls lines 220 & 224 and .tex lines 85 and 89
%  - Deleting the FAQ sidebar entirely: .tex line 188
%  - Removing the Lab/TA Info and placing a brief Overview/About section in
%    their place: uncomment .tex line 91 and .cls line 227, and comment .cls
%    lines for the LAB/TA info that you no longer want (c. lines 184-227)
%
% I am also happy to help with crafting/designing modifications to this
% template to help suite your personal needs in a syllabus. Feel free to
% reach out! 
%
% License:
% The MIT License (see included LICENSE file)
%
%%%%%%%%%%%%%%%%%%%%%%%%%%%%%%%%%%%%%%%%%


%----------------------------------------------------------------------------------------
%	PACKAGES AND OTHER DOCUMENT CONFIGURATIONS
%----------------------------------------------------------------------------------------

\documentclass[a4paper]{inzane_syllabus} % a4paper or letterpaper

\usepackage{booktabs, colortbl, xcolor}
\usepackage{tabularx}
\usepackage{enumitem}
\usepackage{ltablex} 
\usepackage{multirow}

\setlist{nolistsep}

\usepackage{lscape}
\newcolumntype{r}{>{\hsize=0.9\hsize}X}
\newcolumntype{w}{>{\hsize=0.6\hsize}X}
\newcolumntype{m}{>{\hsize=.9\hsize}X}

\renewcommand{\familydefault}{\sfdefault}
\renewcommand{\arraystretch}{2.0}


%--------------------------------------------------------------------------------
%	 PERSONAL INFORMATION
%--------------------------------------------------------------------------------

% Profile picture, if the height of the picture is less than that of the cirle,
% it will have a flat bottom. 
\profilepic{logo.jpeg} 

% To remove any of the following, you need to comment/delete the lines in the
% .cls file (c. line 186). Commenting/deleting the lines below will produce
% an error. 

% To add different lines, you will need to create the new command, e.g.
% \profPhone, in the .cls file c. line 76, and command to create the line in
% the side bar in the .cls file c. line 186
\classname{GEACC} 
\classnum{Projeto POSCOMP 2021} 

%%%%%%%%%%%%%%% PROF INFO
\profname{Não se aplica}
\officehours{}
\office{}
\site{}
\email{}

%%%%%%%%%%%%%%% COURSE  INFO
\prereq{\emph{Pré-requisitos}: inglês e bom aproveitamento (por exemplo: mérito
        acadêmico ou índice de rendimento $\ge 8$) em lógica, cálculo (I, II e
        III), álgebra li-near, matemática discreta, estrutura de dados,
        algoritmos de pesquisa e ordenação, e programação (Java, C/C++ ou
        Python).}
\classdays{A definir}
\classhours{A definir}
\classloc{A definir}

%%%%%%%%%%%%%%% LAB INFO
\labdays{A definir}
\labhours{}
\labloc{}

%%%%%%%%%%%%%%% TA INFO
\taAname{Não se aplica}
\taAofficehours{}
\taAoffice{}
% \taAemail{}
\taBname{}
\taBofficehours{}
\taBoffice{}
% \taBemail{}

% \about{Fish make up the largest group of vertebrates on the planet, easily
% outnumbering mammals, marsupials, birds, and reptiles combined. Not only
% are they abundant, but they've diversified into an extraordinary array of
% sizes, shapes, lifestyles, and habitats. You can find them in the coldest,
% deepest parts of the ocean, and in the hottest freshwater ponds in the desert.
% This course will explore fish diversity and their biology. }


%--------------------------------------------------------------------------------
%	 FAQs
%--------------------------------------------------------------------------------
%to add more questions or remove this section, go to the .cls file and start with
% lines comment lines 226-250. Also comment out this section as well as line
% 152(ish), the command \makeSide

\qOne{Para participar é preciso ter um alto Índice de Rendimento?}
\aOne{Se você está se referindo ao Índice de Rendimento da FAESA, sim. Este
programa de estudos é bem rigoroso e caso seu índice de rendimento esteja abaixo
de 8, acreditamos que você terá muita dificuldade em manter os estudos em dia
(lembre-se: você terá que estudar normalmente para a gradução E para o grupo).}

\qTwo{O cronograma é fixo?}
\aTwo{A princípio sim, mas pode ser alterado a qualquer momento para melhor
atender às necessidades dos participantes.}

\qThree{Eu trabalho/estagio, posso participar?}
\aThree{A princípio sim, desde que você tenha um bom Índice de Rendimento.}

\qFour{Como o grupo funcionará?}
\aFour{Boa pergunta! Ainda estamos definindo isso em maiores detalhes.}


%--------------------------------------------------------------------------------

\begin{document}


%--------------------------------------------------------------------------------
%	 DESCRIPTION
%--------------------------------------------------------------------------------

% Print the sidebar
\makeprofile 


%--------------------------------------------------------------------------------
%	 SOBRE
%--------------------------------------------------------------------------------
% I make liberal use of the \vspace{} command to partition and place sections
% just how I want them. Alter as you see fit.
\section{Sobre o GEACC}

O \emph{Grupo de Estudos Avançados em Ciência da Computação (GEACC)} é
formado por alunos de Ciência da Computação (e cursos relacionados) da
FAESA Centro Universitário.


%--------------------------------------------------------------------------------
%	 OBJETIVO
%--------------------------------------------------------------------------------
\vspace{0.5cm} 
\section{Objetivo do Grupo de Estudos}

Estudar tópicos avançados em Ciência da Computação, em especial:
\begin{itemize}
\item \emph{Matemática}: álgebra linear, análise combinatória, geometria
analítica, ló-gica matemática, matemática discreta, matemática
concreta, probabilidade e estatística
\item \emph{Fundamentos da Computação}: algoritmos e estrutura de dados, análise
de algoritmos, arquitetura e organização de computadores, circuitos di-gitais,
linguagens de programação, linguagens formais, autômatos e computabilidade,
organização de arquivos e dados, sistemas operacionais, técnicas de programação,
teoria dos grafos
\item \emph{Tecnologias da Computação}: banco de dados, compiladores, computação
gráfica, engenharia de software, inteligência artificial, processamento de imagens,
redes de computadores, sistemas distribuídos
\end{itemize}

O objetivo específico é preparar
os alunos participantes do grupo para alcançar nota $\ge 8.0$ na prova POSCOMP
de 2021 (o que corresponde a acertar pelo menos 56 das 70 questões da prova).


%--------------------------------------------------------------------------------
%	 LIMITES
%--------------------------------------------------------------------------------
\vspace{0.5cm}
\section{Limite de Participantes}

O grupo será formado por, no máximo, 5 alunos. Isso é necessário para manter o
padrão e o ritmo de estudos, bem como a qualidade das implementações em código
dos algoritmos e programas criados. Quer participar? acesse a plataforma Piazza
do GEACC para maiores informações (ver abaixo).


%--------------------------------------------------------------------------------
%	 PERMANÊNCIA
%--------------------------------------------------------------------------------
\vspace{0.5cm}
\section{Permanência no Grupo}

Para que um aluno permaneça no grupo, deve participar \emph{ativamente} das
atividades estabelecidas, ou seja: deve cumprir as leituras e tarefas no prazo
indicado, implementar as estruturas de dados e algoritmos que estão sendo
estudados, comparecer aos encontros presenciais e ajudar os outros participantes.
Caso alguém não esteja participando ativamente será excluído e uma nova vaga
será aberta para possíveis interessados.


%--------------------------------------------------------------------------------
%	 FUNCIONAMENTO
%--------------------------------------------------------------------------------
\vspace{0.5cm}
\section{Dinâmica/Funcionamento}

O grupo estabelecerá um cronograma mensal de estudos que será o guia refe-rencial
de tudo o que precisa ser estudado no mês. Esse guia apontará os livros e os
capítulos que precisam ser estudados e ter os exercícios resolvidos (o grupo
decidirá quais exercícios fazer).

Serão realizadas reuniões semanais ou quinzenais, conforme a necessidade, para
que os membros possam discutir sobre o que foi estudado, mostrar os exercícios
realizados e tirar dúvidas.

Além dos encontros presenciais, leituras e estudos adicionais serão realizados
de forma online na plataforma Piazza do grupo.


%--------------------------------------------------------------------------------
%	 PARTICIPAR
%--------------------------------------------------------------------------------
\vspace{0.5cm}
\section{Plataformas Online: Piazza e GitHub}

O GEACC conta com as seguintes plataformas online para auxílio ao estudo:
\begin{itemize}
\item \emph{Piazza}: é o principal "ponto de encontro" online para discussões,
dúvidas, exercícios e tarefas. Acesse em:\\
\url{https://piazza.com/magister.pro.br/winter2019/geacc/home}
\vspace{0.2cm}
\item \emph{GitHub}: contém os documentos e códigos do grupo. Acesse em:\\
\url{https://github.com/geacc}
\end{itemize}


%%%%%%%%%%%%%%%%%%%%%%%%%%%%%%%%%%%%%%%%%%%%%%%%%%%%%%%%%%%%%%%%%%%%%%%%%%%%%%%%%
%                SECOND PAGE
%%%%%%%%%%%%%%%%%%%%%%%%%%%%%%%%%%%%%%%%%%%%%%%%%%%%%%%%%%%%%%%%%%%%%%%%%%%%%%%%%

% Start a new page
\newpage 

% Print the FAQ sidebar
% To get rid of, simply comment out and uncomment \makeFullPage
\makeSide 
%\makeFullPage


%--------------------------------------------------------------------------------
%	 READING MATERIAL
%--------------------------------------------------------------------------------
%\vspace{0.5cm} 
\section{Leituras Obrigatórias}

{\color{myCOLOR} Matemática}\\
Stewart J. \emph{Calculus: Early Transcendentals}. 8th ed. Cengage, 2016. ("Calc")

Graham RL, Knuth DE, Patashnik O. \emph{Concrete Mathematics: a foundation for
computer science}. 2nd ed. Addison-Wesley, 1994. ("MatCon")

Lehman E, Leighton FT, Meyer AR. \emph{Mathematics for Computer Science}. 2018/06/06
ed. \url{https://courses.csail.mit.edu/6.042/spring18/}, 2019. ("MatDis")

Anton H, Rorres C. \emph{Elementary Linear Algebra: applications version}. 10th
ed. John Wiley \& Sons, 2010. ("AlgLin")

Bergmann M, Moor J, Nelson J. \emph{The Logic Book}. 6th ed. McGraw-Hill, 2014
("LogMat")\\

{\color{myCOLOR} Algoritmos}\\
Cormen TH, Leiserson CE, Rivest RL, Stein C. \emph{Introduction to Algorithms}.
3rd ed. The MIT Press, 2009. ("Alg1")

Sedgewick R, Wayne K. \emph{Algorithms}. 4th ed. Addison-Wesley, 2011. ("Alg2")\\

{\color{myCOLOR} Outros}\\
(a definir)


%%%%%%%%%%%%%%%%%%%%%%%%%%%%%%%%%%%%%%%%%%%%%%%%%%%%%%%%%%%%%%%%%%%%%%%%%%%%%
%                COURSE SCHEDULE
%%%%%%%%%%%%%%%%%%%%%%%%%%%%%%%%%%%%%%%%%%%%%%%%%%%%%%%%%%%%%%%%%%%%%%%%%%%%%
\newpage
\makeFullPage
\section{Cronograma de estudos}

\begin{center}
% Change the width of the comments by changing these cm measurements.
% Add/substract columns by adding/deleting p{} sections.
\begin{tabularx}{\textwidth}{p{2cm}p{8cm}p{9.5cm}}  
\arrayrulecolor{myCOLOR}\hline
%%%%%%%%%%%%%%%%%%%%%%%%%%%%%%%%%%%%%%%%%%% MODULE 1
\multicolumn{3}{l}{\textbf{\textcolor{myCOLOR}{\large MÓDULO 1: Base Matemática }}} \\
\hline
% Week & Topic & Readings \\ \hline 
%%Alternatively, instead of Week #, you can do Class date for meeting
Dez/2019 & Functions and Models                      & \emph{Calc},   capítulo 1. \\
         & Limits and Derivatives                    & \emph{Calc},   capítulo 2. \\
         & Recurrent Problems                        & \emph{MatCon}, capítulo 1. \\
         & What is a Proof?                          & \emph{MatDis}, capítulo 1. \\
         & The Well Ordering Principle               & \emph{MatDis}, capítulo 2. \\
         & Systems of Linear Equations and Matrices  & \emph{AlgLin}, capítulo 1. \\

\arrayrulecolor{maingray}\hline
Jan/2020 & Differentiation Rules            & \emph{Calc},   capítulo 3. \\
         & Applications of Differentiation  & \emph{Calc},   capítulo 4. \\
         & Sums                             & \emph{MatCon}, capítulo 2. \\
         & Logical Formulas                 & \emph{MatDis}, capítulo 3. \\
         & Mathematical Data Types          & \emph{MatDis}, capítulo 4. \\
         & Determinants                     & \emph{AlgLin}, capítulo 2. \\

\arrayrulecolor{maingray}\hline
Fev/2020 & Integrals                   & \emph{Calc},   capítulo 5. \\
         & Application of Integration  & \emph{Calc},   capítulo 6. \\
         & Integer Functions           & \emph{MatCon}, capítulo 3. \\
         & Induction                   & \emph{MatDis}, capítulo 5. \\
         & State Machines              & \emph{MatDis}, capítulo 6. \\
         & Euclidean Vectos Spaces     & \emph{AlgLin}, capítulo 3. \\

\arrayrulecolor{maingray}\hline
Mar/2020 & Techniques of Integration            & \emph{Calc},   capítulo 7. \\
         & Further Applications of Integration  & \emph{Calc},   capítulo 8. \\
         & Number Theory                        & \emph{MatCon}, capítulo 4. \\
         & Recursive Data Types                 & \emph{MatDis}, capítulo 7. \\
         & Infinite Sets                        & \emph{MatDis}, capítulo 8. \\
         & General Vector Spaces                & \emph{AlgLin}, capítulo 4. \\

\arrayrulecolor{maingray}\hline
Abr/2020 & Differential Equations                      & \emph{Calc},   capítulo 9.  \\
         & Parametric Equations and Polar Coordinates  & \emph{Calc},   capítulo 10. \\
         & Binomial Coefficients                       & \emph{MatCon}, capítulo 5.  \\
         & Number Theory                               & \emph{MatDis}, capítulo 9.  \\
         & Directed Graphs \& Partial Orders           & \emph{MatDis}, capítulo 10. \\
         & Eigenvalues and Eigenvectors                & \emph{AlgLin}, capítulo 5.  \\

\arrayrulecolor{maingray}\hline
Mai/2020 & Infinite Sequences and Series      & \emph{Calc},   capítulo 11. \\
         & Vectors and the Geometry of Space  & \emph{Calc},   capítulo 12. \\
         & Special Numbers                    & \emph{MatCon}, capítulo 6.  \\
         & Communication Networks             & \emph{MatDis}, capítulo 11. \\
         & Simple Graphs                      & \emph{MatDis}, capítulo 12. \\
         & Inner Product Spaces               & \emph{AlgLin}, capítulo 6.  \\

\arrayrulecolor{maingray}\hline
Jun/2020 & Vector Functions                     & \emph{Calc},   capítulo 13. \\
         & Partial Derivaties                   & \emph{Calc},   capítulo 14. \\
         & Generating Functions                 & \emph{MatCon}, capítulo 7.  \\
         & Planar Graphs                        & \emph{MatDis}, capítulo 13. \\
         & Sums and Asymptotics                 & \emph{MatDis}, capítulo 14. \\
         & Diagonalization and Quadratic Forms  & \emph{AlgLin}, capítulo 7.  \\

\arrayrulecolor{maingray}\hline
Jul/2020 & Multiple Integrals      & \emph{Calc},   capítulo 15. \\
         & Vector Calculus         & \emph{Calc},   capítulo 16. \\
         & Discrete Probability    & \emph{MatCon}, capítulo 8.  \\
         & Cardinality Rules       & \emph{MatDis}, capítulo 15. \\
         & Generating Functions    & \emph{MatDis}, capítulo 16. \\
         & Linear Transformations  & \emph{AlgLin}, capítulo 8.  \\

\arrayrulecolor{maingray}\hline
Ago/2020 & Second-Order Differential Equations  & \emph{Calc},   capítulo 17. \\
         & Asymptotics                          & \emph{MatCon}, capítulo 9.  \\
         & Events and Probability Spaces        & \emph{MatDis}, capítulo 17. \\
         & Conditional Probability              & \emph{MatDis}, capítulo 18. \\
         & Random Variables                     & \emph{MatDis}, capítulo 19. \\
         & Numerical Methods                    & \emph{AlgLin}, capítulo 9.  \\

\arrayrulecolor{maingray}\hline
Set/2020 & Deviation from the Mean         & \emph{MatDis}, capítulo 20. \\
         & Random Walks                    & \emph{MatDis}, capítulo 21. \\
         & Recurrences                     & \emph{MatDis}, capítulo 22. \\
         & Applications of Linear Algebra  & \emph{AlgLin}, capítulo 10. \\

\arrayrulecolor{maingray}\hline
 
         
         
         

\arrayrulecolor{maingray}\hline
 
         

\arrayrulecolor{myCOLOR}\hline
\multicolumn{2}{l}{\textbf{\textcolor{myCOLOR}{\large MÓDULO 2: Algoritmos }}} \\
\hline
Nov/2020 & The Role of Algorithms in Computing  & \emph{Alg1}, capítulo 1.   \\
         & Basic Programming Model              & \emph{Alg2}, capítulo 1.1. \\
         & Data Abstraction                     & \emph{Alg2}, capítulo 1.2. \\

\arrayrulecolor{maingray}\hline
Dez/2020 & Bags, Queues, and Stacks   & \emph{Alg2}, capítulo 1.3. \\
         & Elementary Data Structures & \emph{Alg1}, capítulo 10.  \\
         & Growth of Functions        & \emph{Alg1}, capítulo 3.   \\
         & Analysis of Algorithms     & \emph{Alg2}, capítulo 1.4. \\

\arrayrulecolor{maingray}\hline
Fev/2021 & Case Study: Union-Find    & \emph{Alg2}, capítulo 1.5. \\
         & Elementary Sorts          & \emph{Alg2}, capítulo 2.1. \\
         & Getting Started           & \emph{Alg1}, capítulo 2.   \\
         & Mergesort                 & \emph{Alg2}, capítulo 2.2. \\
         & Divide-and-Conquer        & \emph{Alg1}, capítulo 4.   \\

\arrayrulecolor{maingray}\hline
Mar/2021 & Quicksort    & \emph{Alg1}, capítulo 7. \\
         & Quicksort    & \emph{Alg2}, capítulo 2.3. \\
         & Heapsort     & \emph{Alg1}, capítulo 6. \\
         & Priority Queues & \emph{Alg2}, capítulo 2.4.   \\
         & Applications    & \emph{Alg2}, capítulo 2.5. \\

\arrayrulecolor{maingray}\hline
Abr/2021 & Sorting in Linear Time  & \emph{Alg1}, capítulo 8.   \\
         & Symbol Tables           & \emph{Alg2}, capítulo 3.1. \\
         & Hash Tables             & \emph{Alg2}, capítulo 3.4. \\
         & Hash Tables             & \emph{Alg1}, capítulo 11.  \\
         & Priority Queues         & \emph{Alg2}, capítulo 2.4. \\
         & Applications            & \emph{Alg2}, capítulo 2.5. \\

\arrayrulecolor{maingray}\hline

\arrayrulecolor{maingray}\hline

\arrayrulecolor{maingray}\hline

\arrayrulecolor{myCOLOR}\hline
\multicolumn{2}{l}{\textbf{\textcolor{myCOLOR}{\large MODULE 3: There Goes the Neighborhood }}} \\
\hline
Week 14 & Symbiotic Relationships & DOF Ch. 22, 492-497 \\

& Behavior & DOF Ch. 23 \\
\arrayrulecolor{maingray}\hline
Week 15 & Ecology & DOF Ch. 25 \\

& Conservation Efforts & DOF Ch. 26 \\
\arrayrulecolor{myCOLOR}\hline
Week 16 & FINAL EXAM & Date \& Time \& Location \\ 
\hline 
\end{tabularx}
\end{center}

%%%%%%%%%%%%%%%%%%%%%%%%%%%%%%%%%%%%%%%%%%%%%%%%%%%%%%%%%%%%%%%%%%%%%%%%%%%%%
%                LAB SCHEDULE
%%%%%%%%%%%%%%%%%%%%%%%%%%%%%%%%%%%%%%%%%%%%%%%%%%%%%%%%%%%%%%%%%%%%%%%%%%%%%
\newpage
\section{Lab Schedule}

\begin{center}
\begin{tabularx}{\textwidth}{p{2cm}p{6.5cm}p{11cm}}
\arrayrulecolor{myCOLOR}\hline
Week 2 & Chondrichthyan Fishes & Students enjoy a two part lab: first, they examine specimens across the Chondrichthyan phylogeny; second, they dissect a small spiny dogfish shark. \\
\arrayrulecolor{maingray}\hline 
Week 3 & Harvard Natural History Museum & Students walk through the HMNH and the fossil collection, inspecting various fossil fishes. \\
\hline 
Week 4 & Basal Teleosts \& Otocephalan Fishes & Students explore specimens across the basal Teleost phylogeny. \\
\hline 
Week 5 & Freshwater \& Deep-Sea Fishes & Students explore specimens from a diverse group of fishes, and try to place each group in the broader phylogeny. \\
\hline 
Week 6 & Coral Reef \& Pelagic Fishes & Students explore specimens from a diverse group of fishes, and try to place each group in the broader phylogeny.\\
\hline 
Week 7 & No Lab & \\ 
\hline 
Week 8 & Internal Systems & Students dissect fish specimens, probing and examing key internal systems. \\
\hline 
Week 9 & Jaw Dissections & Students again dissect their fish specimens, taking apart and visualizing the jaws of their fish. \\
\hline 
Week 10 & Sensory Systems \& Buoyancy & Students again enjoy a two-part lab: first, examining a broad selection of specimens, comparing and contrasting sensory system apparatuses; and then conducting a series of small experiments to better understand the difficulties associated with buoyancy control in the water. \\
\hline 
Week 11 & Locomotion & Students dissect fish specimens, looking at muscular and structure of the body and fins. Students also participate in demonstrations designed to elucidate the concept of lift. \\
\hline 
Week 12 & Review Paper Projects & Students bring electronic devices and/or paper printouts of 2-3 paper choices, and will select peer reviewers. TAs will be available to assist students in choosing a paper and begin reviewing it. \\
\hline 
Week 13 & No Lab & \\
\hline 
Week 14 & No Lab & \\
\hline 
Week 15 & Final Exam Review Sessions & Review Paper Project Due \\
\arrayrulecolor{myCOLOR}\hline

\end{tabularx}
\end{center}

%----------------------------------------------------------------------------------------

\end{document}
