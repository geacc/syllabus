%%%%%%%%%%%%%%%%%%%%%%%%%%%%%%%%%%%%%%%%%
% Inzane Syllabus Template
% LaTeX Template
% Version 1.2 (8.22.2019)
%
% This template has been downloaded from:
% http://www.LaTeXTemplates.com
%
% Original author:
% Carmine Spagnuolo (cspagnuolo@unisa.it) with major modifications by 
% Zane Wolf (zwolf.mlxvi@gmail.com)
%
% I (Zane) have left a lot of instructions both in the .tex file and
% the .cls file that can guide you to customize this document to suite
% your tastes and requirements. Here is a brief guide: 
%  - Changing the Main Color: .cls line 39
%  - Adding more FAQs: .cls line 126 and .tex line 99
%  - Adding TA emails: uncomment .cls lines 220 & 224 and .tex lines 85 and 89
%  - Deleting the FAQ sidebar entirely: .tex line 188
%  - Removing the Lab/TA Info and placing a brief Overview/About section in
%    their place: uncomment .tex line 91 and .cls line 227, and comment .cls
%    lines for the LAB/TA info that you no longer want (c. lines 184-227)
%
% I am also happy to help with crafting/designing modifications to this
% template to help suite your personal needs in a syllabus. Feel free to
% reach out! 
%
% License:
% The MIT License (see included LICENSE file)
%
%%%%%%%%%%%%%%%%%%%%%%%%%%%%%%%%%%%%%%%%%


%----------------------------------------------------------------------------------------
%	PACKAGES AND OTHER DOCUMENT CONFIGURATIONS
%----------------------------------------------------------------------------------------

\documentclass[a4paper]{inzane_syllabus} % a4paper or letterpaper

\usepackage{booktabs, colortbl, xcolor}
\usepackage{tabularx}
\usepackage{enumitem}
\usepackage{ltablex} 
\usepackage{multirow}

\setlist{nolistsep}

\usepackage{lscape}
\newcolumntype{r}{>{\hsize=0.9\hsize}X}
\newcolumntype{w}{>{\hsize=0.6\hsize}X}
\newcolumntype{m}{>{\hsize=.9\hsize}X}

\renewcommand{\familydefault}{\sfdefault}
\renewcommand{\arraystretch}{2.0}


%--------------------------------------------------------------------------------
%	 PERSONAL INFORMATION
%--------------------------------------------------------------------------------

% Profile picture, if the height of the picture is less than that of the cirle,
% it will have a flat bottom. 
\profilepic{logo.jpeg} 

% To remove any of the following, you need to comment/delete the lines in the
% .cls file (c. line 186). Commenting/deleting the lines below will produce
% an error. 

% To add different lines, you will need to create the new command, e.g.
% \profPhone, in the .cls file c. line 76, and command to create the line in
% the side bar in the .cls file c. line 186
\classname{GEACC} 
\classnum{Projeto POSCOMP 2021} 

%%%%%%%%%%%%%%% PROF INFO
\profname{Abrantes Araújo Silva Filho}
\officehours{Office Hrs: Mon \& Wed 1-2p} 
\office{MCZ Labs 105}
\site{http://inzaneresearch.com} 
\email{rzwolf@g.harvard.com}

%%%%%%%%%%%%%%% COURSE  INFO
\prereq{\emph{Pré-requisitos}: inglês e bom aproveitamento (por exemplo: mérito
        acadêmico ou índice de rendimento $> 7$) em lógica, cálculo (I, II e
        III), álgebra li-near, matemática discreta, estrutura de dados,
        algoritmos de pesquisa e ordenação, e programação (Java, C/C++ ou
        Python).}
\classdays{A definir}
\classhours{A definir}
\classloc{A definir}

%%%%%%%%%%%%%%% LAB INFO
\labdays{}
\labhours{}
\labloc{}

%%%%%%%%%%%%%%% TA INFO
\taAname{}
\taAofficehours{}
\taAoffice{}
% \taAemail{}
\taBname{}
\taBofficehours{}
\taBoffice{}
% \taBemail{}

% \about{Fish make up the largest group of vertebrates on the planet, easily
% outnumbering mammals, marsupials, birds, and reptiles combined. Not only
% are they abundant, but they've diversified into an extraordinary array of
% sizes, shapes, lifestyles, and habitats. You can find them in the coldest,
% deepest parts of the ocean, and in the hottest freshwater ponds in the desert.
% This course will explore fish diversity and their biology. }


%--------------------------------------------------------------------------------
%	 FAQs
%--------------------------------------------------------------------------------
%to add more questions or remove this section, go to the .cls file and start with
% lines comment lines 226-250. Also comment out this section as well as line
% 152(ish), the command \makeSide

\qOne{Para participar é preciso ter um alto Índice de Rendimento?}
\aOne{Se você está se referindo ao Índice de Rendimento da FAESA, sim. Este
programa de estudos é bem rigoroso e caso seu índice de rendimento esteja abaixo
de 7, acreditamos que você terá muita dificuldade em manter os estudos em dia
(lembre-se: você terá que estudar para a gradução E TAMBÉM para o grupo).}

\qTwo{O cronograma é fixo?}
\aTwo{A princípio sim, mas pode ser alterado a qualquer momento para melhor
atender às necessidades dos participantes.}

\qThree{Eu trabalho/estagio, posso participar?}
\aThree{A princípio sim, desde que você tenha um bom Índice de Rendimento.}

\qFour{Como o grupo funcionará?}
\aFour{Boa pergunta! Ainda estamos definindo isso em maiores detalhes.}


%--------------------------------------------------------------------------------

\begin{document}


%--------------------------------------------------------------------------------
%	 DESCRIPTION
%--------------------------------------------------------------------------------

% Print the sidebar
\makeprofile 


%--------------------------------------------------------------------------------
%	 SOBRE
%--------------------------------------------------------------------------------
\section{Sobre o GEACC}

O \emph{Grupo de Estudos Avançados em Ciência da Computação (GEACC)} é
formado por alunos de Ciência da Computação (ou outra área relacionada) da
FAESA Centro Universitário.

%--------------------------------------------------------------------------------
%	 OBJETIVO
%--------------------------------------------------------------------------------
\vspace{0.5cm} 
\section{Objetivo}

Estudar tópicos avançados em Ciência da Computação, principalmente relacionados
a \emph{matemática} (contínua, discreta e concreta), \emph{algoritmos e
estruturas de dados} (fundamentais e avançados) e \emph{aplicações} (científicas,
data science, machine learning e outras), com o objetivo específico de preparar
os alunos participantes para alcançar nota $\ge 8.5$ na prova POSCOMP de 2021 (o
que corresponde a acertar pelo menos 60 das 70 questões da prova).



%--------------------------------------------------------------------------------
%	 READING MATERIAL
%--------------------------------------------------------------------------------
% I make liberal use of the \vspace{} command to partition and place sections
% just how I want them. Alter as you see fit.

\vspace{0.5cm} 
\section{Material}

{\color{myCOLOR} Livros Obrigatórios}\\
Cormen TH, Leiserson CE, Rivest RL, Stein C. \emph{Introduction to Algorithms}.
3 ed. Cambridge, 2009 ("Alg1") \\

{\color{myCOLOR} Recommended Text}\\
Paxton, J.R. \& Eschmeyer, W.N. \textit{Encyclopedia of Fishes}. 2nd Edition.
Harcourt Brace \& Co. 1998. \\

{\color{myCOLOR} Other}\\
Any required journal articles and book chapters will be provided on Canvas.


%--------------------------------------------------------------------------------
%	 GRADING SCHEME
%--------------------------------------------------------------------------------
\vspace{0.5cm}
\section{Grading Scheme}

% Below is the \twentyshort environment - a list with only two inputs. However,
% there is a \twenty environment, which creates a list with four inputs. You can
% find/alter details of that table in the .cls file c. lines 320. 
\begin{twentyshort}
	%\twentyitemshort{X\%}{Attendance/Participation}
	\twentyitemshort{15\%}{Review Paper}
	\twentyitemshort{15\%}{Lab Worksheets}
    \twentyitemshort{40\%}{Midterm Exams, 20\% each}
    \twentyitemshort{30\%}{Final Exam}
\end{twentyshort}

Grades will follow the standard scale: A = 89.5-100; B = 79.5-89.4; C = 69.5-79.4;
D = 60-69.4; F $<$60. Curving is at the discretion of the professor. 


%--------------------------------------------------------------------------------
%	 EXTRAS
%--------------------------------------------------------------------------------

\vspace{0.5cm}
\section{Review Paper}

Students will choose a scientific article concerning a topic or species that we
covered in class. For this assignment, you will write a summary of the paper and
a review: strengths of the paper, things they could improve, perhaps any holes
that they did not address, etc. You will then give your review to two classmates
to independently review, and you will incorporate their edits into your final
draft. You will turn in an abstract of the original paper, the two peer-reviewed
copies of your review, the names of people whose papers you reviewed, and your
final draft. 15\% of your grade will depend on how thoughtfully and thoroughly
you reviewed your peers' papers.   

\vspace{0.5cm}
\section{Learning Objectives}

% use \begin{outline} or \begin{outline}[enumerate] to create a list with subitems. 
\begin{itemize}
\item Become familiar with the evolutionary history and taxonomic diversity
  of fishes
\item Improve your understanding of the basic physiological and behavioral
  adaptations of fishes
\item Gain skills regarding the dissection, collection, and preservation of fish
  specimens through laboratory work
\item Learn to critically review a paper and summarize it, as well as review and
  provide helpful criticism to your peers' work
\end{itemize}


%%%%%%%%%%%%%%%%%%%%%%%%%%%%%%%%%%%%%%%%%%%%%%%%%%%%%%%%%%%%%%%%%%%%%%%%%%%%%%%%%
%                SECOND PAGE
%%%%%%%%%%%%%%%%%%%%%%%%%%%%%%%%%%%%%%%%%%%%%%%%%%%%%%%%%%%%%%%%%%%%%%%%%%%%%%%%%

% Start a new page
\newpage 

% Print the FAQ sidebar
% To get rid of, simply comment out and uncomment \makeFullPage
\makeSide 
%\makeFullPage

\vspace{0.5cm}
\section{Make-up Policy}

Make-up exams or assignments will only be allowed for students who have a
substantiated excuse approved by the instructor \emph{before the due date}.
Leaving a phone message or sending an e-mail without confirmation is not
acceptable. Labs are mandatory. Make-ups for missing a lab consists of a 1
paragraph summary of a recent fish-oriented journal article highlighted in the
news AND a 4 minute power point presentation on the article to the class. Any
additional missed labs will result in zero credit for that lab.

\vspace{0.5cm}
\section{Diversity and Inclusivity Statement}

I consider this classroom to be a place where you will be treated with respect,
and I welcome individuals of all ages, backgrounds, beliefs, ethnicities,
genders, gender identities, gender expressions, national origins, religious
affiliations, sexual orientations, ability - and other visible and non-visible
differences. All members of this class are expected to contribute to a
respectful, welcoming and inclusive environment for every other member of
the class. 

\vspace{0.5cm}
\section{Accommodations for Students with Disabilities}

If you are a student with learning needs that require special accommodation,
contact the Office of Disability Services at 555-5555 or theiremail@email.com,
as soon as possible, to make an appointment to discuss your special needs and
to obtain an accommodations letter.  Please e-mail me as soon as possible in
order to set up a time to discuss your learning needs.

\vspace{0.5cm}
\section{Academic Integrity}

The University Code of Academic Integrity is central to the ideals of this
course. Students are expected to be independently familiar with the Code and to
recognize that their work in the course is to be their own original work that
truthfully represents the time and effort applied.  Violations of the Code are
most serious and will be handled in a manner that fully represents the extent
of the Code and that befits the seriousness of its violation.\\


%%%%%%%%%%%%%%%%%%%%%%%%%%%%%%%%%%%%%%%%%%%%%%%%%%%%%%%%%%%%%%%%%%%%%%%%%%%%%
%                COURSE SCHEDULE
%%%%%%%%%%%%%%%%%%%%%%%%%%%%%%%%%%%%%%%%%%%%%%%%%%%%%%%%%%%%%%%%%%%%%%%%%%%%%
\newpage
\makeFullPage
\section{Cronograma de estudos}

\begin{center}
\begin{tabularx}{\textwidth}{p{2cm}p{8cm}p{9.5cm}} %change the width of the comments by changing these cm measurements. Add/substract columns by adding/deleting p{} sections. 
\arrayrulecolor{myCOLOR}\hline
%%%%%%%%%%%%%%%%%%%%%%%%%%%%%%%%%%%%%%%%%%% MODULE 1
\multicolumn{3}{l}{\textbf{\textcolor{myCOLOR}{\large MÓDULO 1: Base Matemática }}} \\
\hline
% Week & Topic & Readings \\ \hline 
%%Alternatively, instead of Week #, you can do Class date for meeting
Dez/2019 & Functions and Models                     & \emph{Calc},   capítulo 1. \\
         & Recurrent Problems                       & \emph{MatCon}, capítulo 1. \\
         & What is a Proof?                         & \emph{MatDis}, capítulo 1. \\
         & The Well Ordering Principle              & \emph{MatDis}, capítulo 2. \\
         & Systems of Linear Equations and Matrices & \emph{AlgLin}, capítulo 1. \\

\arrayrulecolor{maingray}\hline
Jan/2020 & Limits and Derivatives      & \emph{Calc},   capítulo 2. \\
         & Sums                        & \emph{MatCon}, capítulo 2. \\
         & Logical Formulas            & \emph{MatDis}, capítulo 3. \\
         & Mathematical Data Types     & \emph{MatDis}, capítulo 4. \\
         & Determinants                & \emph{AlgLin}, capítulo 2. \\

\arrayrulecolor{maingray}\hline
Fev/2020 & Differentiation Rules       & \emph{Calc},   capítulo 3. \\
         & Integer Functions           & \emph{MatCon}, capítulo 3. \\
         & Induction                   & \emph{MatDis}, capítulo 5. \\
         & Recursive Data Types        & \emph{MatDis}, capítulo 6. \\
         & Euclidean Vectos Spaces     & \emph{AlgLin}, capítulo 3. \\

\arrayrulecolor{maingray}\hline
Mar/2020 & Applications of Differentiation  & \emph{Calc},   capítulo 4. \\
         & Number Theory                    & \emph{MatCon}, capítulo 4. \\
         & Infinite Sets                    & \emph{MatDis}, capítulo 7. \\
         & Number Theory                    & \emph{MatDis}, capítulo 8. \\
         & General Vector Spaces            & \emph{AlgLin}, capítulo 4. \\

\arrayrulecolor{maingray}\hline
Abr/2020 & Integrals                          & \emph{Calc},   capítulo 5.  \\
         & Binomial Coefficients              & \emph{MatCon}, capítulo 5.  \\
         & Directed Graphs \& Partial Orders  & \emph{MatDis}, capítulo 9.  \\
         & Communication Networks             & \emph{MatDis}, capítulo 10. \\
         & Eigenvalues and Eigenvectors       & \emph{AlgLin}, capítulo 5.  \\

\arrayrulecolor{maingray}\hline
Mai/2020 & Application of Integration  & \emph{Calc},   capítulo 6.  \\
         & Special Numbers             & \emph{MatCon}, capítulo 6.  \\
         & Simple Graphs               & \emph{MatDis}, capítulo 11. \\
         & Planar Graphs               & \emph{MatDis}, capítulo 12. \\
         & Inner Product Spaces        & \emph{AlgLin}, capítulo 6.  \\

\arrayrulecolor{maingray}\hline
Jun/2020 & Techniques of Integration            & \emph{Calc},   capítulo 7.  \\
         & Generating Functions                 & \emph{MatCon}, capítulo 7.  \\
         & Sums and Asymptotics                 & \emph{MatDis}, capítulo 13. \\
         & Cardinality Rules                    & \emph{MatDis}, capítulo 14. \\
         & Diagonalization and Quadratic Forms  & \emph{AlgLin}, capítulo 7.  \\

\arrayrulecolor{maingray}\hline
Jul/2020 & Further Applications of Integration  & \emph{Calc},   capítulo 8.  \\
         & Discrete Probability                 & \emph{MatCon}, capítulo 8.  \\
         & Generating Functions                 & \emph{MatDis}, capítulo 15. \\
         & Events and Probability Spaces        & \emph{MatDis}, capítulo 16. \\
         & Linear Transformations               & \emph{AlgLin}, capítulo 8.  \\

\arrayrulecolor{maingray}\hline
Ago/2020 & Differential Equations   & \emph{Calc},   capítulo 9.  \\
         & Asymptotics              & \emph{MatCon}, capítulo 9.  \\
         & Conditional Probability  & \emph{MatDis}, capítulo 17. \\
         & Random Variables         & \emph{MatDis}, capítulo 18. \\
         & Numerical Methods        & \emph{AlgLin}, capítulo 9.  \\

\arrayrulecolor{maingray}\hline
Set/2020 & Parametric Equations and Polar Coordinates  & \emph{Calc},   capítulo 10. \\
         & Deviation from the Mean                     & \emph{MatDis}, capítulo 19. \\
         & Random Walks                                & \emph{MatDis}, capítulo 20. \\
         & Recurrences                                 & \emph{MatDis}, capítulo 21. \\
         & Applications of Linear Algebra              & \emph{AlgLin}, capítulo 10. \\

\arrayrulecolor{maingray}\hline
Out/2020 & Infinite Sequences and Series      & \emph{Calc}, capítulo 11. \\
         & Vectors and the Geometry of Space  & \emph{Calc}, capítulo 12. \\
         & Vector Functions                   & \emph{Calc}, capítulo 13. \\
         & Partial Derivaties                 & \emph{Calc}, capítulo 14. \\
         & Multiple Integrals                 & \emph{Calc}, capítulo 15. \\

\arrayrulecolor{maingray}\hline
Nov/2020 & Vector Calculus                      & \emph{Calc}, capítulo 16. \\
         & Second-Order Differential Equations  & \emph{Calc}, capítulo 17. \\

\arrayrulecolor{myCOLOR}\hline
\multicolumn{2}{l}{\textbf{\textcolor{myCOLOR}{\large MÓDULO 2: Algoritmos }}} \\
\hline
Nov/2020 & The Role of Algorithms in Computing  & \emph{Alg1}, capítulo 1.   \\
         & Basic Programming Model              & \emph{Alg2}, capítulo 1.1. \\
         & Data Abstraction                     & \emph{Alg2}, capítulo 1.2. \\

\arrayrulecolor{maingray}\hline
Dez/2020 & Bags, Queues, and Stacks   & \emph{Alg2}, capítulo 1.3. \\
         & Elementary Data Structures & \emph{Alg1}, capítulo 10.  \\
         & Growth of Functions        & \emph{Alg1}, capítulo 3.   \\
         & Analysis of Algorithms     & \emph{Alg2}, capítulo 1.4. \\

\arrayrulecolor{maingray}\hline
Fev/2021 & Case Study: Union-Find    & \emph{Alg2}, capítulo 1.5. \\
         & Elementary Sorts          & \emph{Alg2}, capítulo 2.1. \\
         & Getting Started           & \emph{Alg1}, capítulo 2.   \\
         & Mergesort                 & \emph{Alg2}, capítulo 2.2. \\
         & Divide-and-Conquer        & \emph{Alg1}, capítulo 4.   \\

\arrayrulecolor{maingray}\hline
Mar/2021 & Quicksort    & \emph{Alg1}, capítulo 7. \\
         & Quicksort    & \emph{Alg2}, capítulo 2.3. \\
         & Heapsort     & \emph{Alg1}, capítulo 6. \\
         & Priority Queues & \emph{Alg2}, capítulo 2.4.   \\
         & Applications    & \emph{Alg2}, capítulo 2.5. \\

\arrayrulecolor{maingray}\hline
Abr/2021 & Sorting in Linear Time  & \emph{Alg1}, capítulo 8.   \\
         & Symbol Tables           & \emph{Alg2}, capítulo 3.1. \\
         & Hash Tables             & \emph{Alg2}, capítulo 3.4. \\
         & Hash Tables             & \emph{Alg1}, capítulo 11.  \\
         & Priority Queues         & \emph{Alg2}, capítulo 2.4. \\
         & Applications            & \emph{Alg2}, capítulo 2.5. \\

\arrayrulecolor{maingray}\hline

\arrayrulecolor{maingray}\hline

\arrayrulecolor{maingray}\hline

\arrayrulecolor{myCOLOR}\hline
\multicolumn{2}{l}{\textbf{\textcolor{myCOLOR}{\large MODULE 3: There Goes the Neighborhood }}} \\
\hline
Week 14 & Symbiotic Relationships & DOF Ch. 22, 492-497 \\

& Behavior & DOF Ch. 23 \\
\arrayrulecolor{maingray}\hline
Week 15 & Ecology & DOF Ch. 25 \\

& Conservation Efforts & DOF Ch. 26 \\
\arrayrulecolor{myCOLOR}\hline
Week 16 & FINAL EXAM & Date \& Time \& Location \\ 
\hline 
\end{tabularx}
\end{center}

%%%%%%%%%%%%%%%%%%%%%%%%%%%%%%%%%%%%%%%%%%%%%%%%%%%%%%%%%%%%%%%%%%%%%%%%%%%%%
%                LAB SCHEDULE
%%%%%%%%%%%%%%%%%%%%%%%%%%%%%%%%%%%%%%%%%%%%%%%%%%%%%%%%%%%%%%%%%%%%%%%%%%%%%
\newpage
\section{Lab Schedule}

\begin{center}
\begin{tabularx}{\textwidth}{p{2cm}p{6.5cm}p{11cm}}
\arrayrulecolor{myCOLOR}\hline
Week 2 & Chondrichthyan Fishes & Students enjoy a two part lab: first, they examine specimens across the Chondrichthyan phylogeny; second, they dissect a small spiny dogfish shark. \\
\arrayrulecolor{maingray}\hline 
Week 3 & Harvard Natural History Museum & Students walk through the HMNH and the fossil collection, inspecting various fossil fishes. \\
\hline 
Week 4 & Basal Teleosts \& Otocephalan Fishes & Students explore specimens across the basal Teleost phylogeny. \\
\hline 
Week 5 & Freshwater \& Deep-Sea Fishes & Students explore specimens from a diverse group of fishes, and try to place each group in the broader phylogeny. \\
\hline 
Week 6 & Coral Reef \& Pelagic Fishes & Students explore specimens from a diverse group of fishes, and try to place each group in the broader phylogeny.\\
\hline 
Week 7 & No Lab & \\ 
\hline 
Week 8 & Internal Systems & Students dissect fish specimens, probing and examing key internal systems. \\
\hline 
Week 9 & Jaw Dissections & Students again dissect their fish specimens, taking apart and visualizing the jaws of their fish. \\
\hline 
Week 10 & Sensory Systems \& Buoyancy & Students again enjoy a two-part lab: first, examining a broad selection of specimens, comparing and contrasting sensory system apparatuses; and then conducting a series of small experiments to better understand the difficulties associated with buoyancy control in the water. \\
\hline 
Week 11 & Locomotion & Students dissect fish specimens, looking at muscular and structure of the body and fins. Students also participate in demonstrations designed to elucidate the concept of lift. \\
\hline 
Week 12 & Review Paper Projects & Students bring electronic devices and/or paper printouts of 2-3 paper choices, and will select peer reviewers. TAs will be available to assist students in choosing a paper and begin reviewing it. \\
\hline 
Week 13 & No Lab & \\
\hline 
Week 14 & No Lab & \\
\hline 
Week 15 & Final Exam Review Sessions & Review Paper Project Due \\
\arrayrulecolor{myCOLOR}\hline

\end{tabularx}
\end{center}

%----------------------------------------------------------------------------------------

\end{document}
